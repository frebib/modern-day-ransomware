\documentclass[10pt,a4paper]{article}
\usepackage[margin=9em]{geometry}
\usepackage[utf8]{inputenc}
\usepackage[UKenglish]{babel}
\usepackage[UKenglish]{isodate}
\usepackage{eurosym}
\usepackage[hyphens]{url}

% Fix paragraph spacing and use of noindent
\usepackage[parfill]{parskip}
\usepackage{titlesec}
\titlespacing{\section}{0pt}{1.4\parskip}{0.2\parskip}
\titlespacing{\subsection}{0pt}{1.4\parskip}{0.2\parskip}
\titlespacing{\subsubsection}{0pt}{1.4\parskip}{0.2\parskip}

\begin{document}

\begin{titlepage}
  \vspace*{\fill}
  \begin{center}
    {\Huge \textbf{Modern Day Ransomware}}\\[0.5cm]
    {\Large Edouard Andrei, Matthias Casula,}\\
    {\Large Joe Groocock \& Kwun Lau}\\[0.6cm]
    {\large Secure Programming}\\
    {\large University of Birmingham}\\[0.4cm]
    {\large \printdate{2017-11-19}}
  \end{center}
  \vspace*{\fill}
\end{titlepage}

\tableofcontents

\newpage

\section{Introduction}
\subsection{Description of the problem}
Ransomware is a type of malware that prevents or limits users from accessing their system, either by locking the system's screen or by locking the users' files until a ransom is paid. More modern ransomware families, collectively categorized as crypto-ransomware, encrypt certain file types on infected systems and force users to pay the ransom through given online payment methods to get a decrypt key.\cite{trend}

Starting from around 2012 the use of ransomware frauds has grown worldwide.\cite{grown1}\cite{grown2} In June 2013, vendor McAfee released data showing that it had collected more than double the number of samples of ransomware that quarter than it had in the same quarter of the previous year.\cite{mcafee} CryptoLocker was particularly successful, procuring an estimated US \$3 million before it was taken down by authorities\cite{bbc}, and CryptoWall was estimated by the US Federal Bureau of Investigation (FBI) to have accrued over US \$18m by June 2015.\cite{cyberwall}

\textbf{Symantec has classified ransomware to be the most dangerous cyber threat. \cite{symantec}}

\subsection{Motivation}
Traditionally, crypto-ransomware targeted individuals and encrypted their personal data and files as a form of extortion for hundreds of dollars. Ransomware has evolved to target businesses and government agencies for much larger financial gains. Ransomware authors are pivoting their attacks from individuals to government entities and health care institutions, causing a threat to public safety. Security is at the heart of everything, it's what protects our privacy and that's why it's so important. Whenever there is an attack targeted towards a system you have to find a fast and efficient way to combat it. In order to do that we need to understand how it happened and need to learn how we can prevent such attacks from happening.

\section{Ransomware attacks}

\subsection{Reveton}
Reveton (AKA Police Trojan) is a variant in the family of ransomware Trojan that targeted European users on the Windows operating systems (Windows XP \& Windows 7) since 2012. This ransomware can infect computers only by drive-by downloads, which means the Trojan files must be downloaded to the system via some sort of sources such as spam websites or e-mail. For example, an attacker can attach a Trojan file to a e-mail and send it to numerous of contacts, where the context of the e-mail or the name of the attachment file can be anything that might lure the victims to download it.

Once Reveton successfully infects the computer, it will prevent the victim from accessing the system, and display a message mimicking from a local law enforcement authority warning that illegal content is being stored in the victim’s computer. It also states that a fine must be paid to avoid prosecution.
Upon execution, it follows these steps\cite{revetonanalysis}:

\begin{enumerate}
\item It copies itself to a non-suspicious-looking folder
\item Creates a registry entry in the autorun registry hive
\item Runs in special installation mode – injects installation code to \texttt{explorer.exe} that connects to its command-and-control server
\item Download the local authority’s logo and local language that matches the geometric information of victim’s IP address.
\item When running in ransomware mode, it creates two threads:
\begin{itemize}
\item \textbf{\textit{Thread 1: }}An endless loop that displays a picture of a warning message and a local authority’s logo, and waits for the victim to enter his/her payment information.
\item \textbf{\textit{Thread 2: }}Another endless loop that constantly checks for specific processes which may be able to stop or terminate the execution of Reveton, and if any is found, it would kill the process immediately.
\end{itemize}
\end{enumerate}

\subsection{CryptoLocker}
Between September 2013 and May 2014, CryptoLocker, a trojan-based ransomware was distributed via legitimate looking infected email attachments and also the \textit{Gameover ZeuS} botnet infecting computers running Windows. Upon receipt of the infected ZIP file, the user would extract it and unknowingly run an executable file disguised as a PDF displayed with the same icon used by PDF files -- this was also assisted by Windows hiding file extensions from the user.by default.

Upon installation this attack embedded itself as system service so that it would always be run when the computer was restarted. Attempts are then made to contact one or many \textit{Command and Control} servers, which can be a local or remote server which varied on a frequent basis to reduce traceability. Once connected, the server generated a 2048 bit RSA key and sent the public component to the infected computer to be used for encryption.

CryptoLocker targeted files with particular extensions such as office documents, photographs and AutoCAD drawing files, encrypting them in-place with the unique public key before displaying a window with a demand of \$400 USD, \euro400 or the equivalent in BitCoin (BTC), starting and 2 BTC, adjusted down to just 0.3 BTC with the fluctuating price of BitCoin. Upon payment the user would be sent a link to download a binary preloaded with the appropriate private key that could be used to decrypt all encrypted files however there were reports of some users not receiving a decryption key even after paying the ransom.

Around mid-2014, CryptoLocker was mostly stopped due to many law enforcement agencies, universities and software vendors across the world interrupting the operation and spread of the \textit{GameOver ZeuS} botnet.

\subsection{WannaCry}
WannaCry (AKA WannaCrypt0r), is piece of ransomware that had a worldwide outburst in May 2017 and has affected major companies such as the British National Health Service (NHS), the Spanish company Telefónica and the German railway company Deutsche Bahn, causing major disruptions. The behavior of it consists in encrypting the data on the machine using RSA 2048, a strong asymmetric encryption method and demanding payment of \$300 worth of bitcoin currency in order to recover the data. The malicious software targets MS Windows systems making use of two exploits developed by the American National Security Agency (NSA) that were leaked on Tuesday 14th of April 2017 by the hacker group The Shadow Brokers. The two exploits are EternalBlue and Double Pulsar which are completely independent of each other.

\textbf{\textit{EternalBlue}} exploits a vulnerability in Microsoft's implementation of Server Message Block (SMBv1), a protocol commonly used for sharing access to files, printers, serial ports and other services across a network. Such vulnerability consists in erroneous processing of packets carefully crafted by a remote attacker.

\textbf{\textit{Double Pulsar}} is a backdoor tool used as the primary payload by recent exploits developed by the NSA. It runs in privileged Kernel mode, making use of 3 commands: \texttt{ping}, \texttt{kill} and \texttt{exec}. The latter is used to run additional malware on the infected system. The backdoor communicates with remote sources over port 445 (used for SMB over IP), producing a distinctive response by which it can be detected, when prompted with a correctly crafted request.

In order to anonymize communication to its designers, WannaCrypt0r makes use of \textit{\textbf{Tor}}, which is installed when the system is infected. Tor provides a Hidden Service Protocol \cite{torcircuits} which makes it extremely difficult to localize the malicious remote service being run by the attackers.
Fortunately, the widely spread versions of WannaCry were found to have a kill switch after in depth analysis by Marcus Hutchins, a 22 year old English security researcher. This switch consisted in a URL  embedded in the ransomware executable that, if reachable, prevented the malware from deploying its payload and consequently spreading. Some of the later distributions of the malware seem to not have a kill switch. Contrarily to common belief, most of the infected systems were running a 64bit distribution of Windows 7, not Windows XP, which had a minimal impact in the spread of the attack \cite{wanacrySystems}.

\section{Mitigations} \label{sec:mitigations}

\subsection{Reveton}

\subsubsection{Trojan Removal}
When a computer is infected by Reveton, the trojan files are ran at system startup. Therefore by booting Windows system to safe mode can avoid activation of Reveton, and victim can then remove Trojan related files\cite{revetonremoval}. The Trojan file location varies Depending on the version of Reveton.

\subsection{CryptoLocker}

\subsection{WannaCry}

\subsubsection{Patches}
Microsoft was made aware by the NSA of the possible theft of EternalBlue and had developed and released critical security updates for all their supported operating systems at the time, and additionally the recently discontinued Vista in March 2017, 2 months prior to the spread of WannaCry.
Additionally, a day after the outbreak, the software house released a one off security patch for their discontinued systems, XP and Server 2003.
The impact of the malware could have mitigated by simply keeping up to date with critical updates. In environments where downtime needs to be kept to a bare minimum this can be difficult but, as proven by the outburst of WannaCry, failure to patch the systems can result in much more inconvenient disruptions than those caused by a scheduled update.

\subsubsection{Register Kill switch domain}
The aforementioned kill switch can be found relatively easily conducting some static analysis on the portable executable of WannaCry, by inspecting it with \texttt{pestr} which is part of the ``PEV'' toolkit for file analysis \cite{PEtoolkit}. By running the \texttt{pestr --net <wanacryExecName>} command, one of the strings returned will be the kill switch URL \cite{pestrVideo}. Registering such domain will prevent the malware from executing and spreading further.

\subsubsection{Blacklisting related IP addresses}
By conducting some behavioral analysis conducted on the ``WannaDecrypt0r'' executable, string generated during its execution can be observed and a RegEx filter applied to isolate all strings listing IPv4 addresses as demonstrated in \cite{wanacryFindIpAddr}. These are the addresses that will be used in the setup of the Tor circuit used to remotely communicate with the attackers. Such addresses could be blacklisted in the firewall to prevent further communication.

\subsection{General good practice}
These are some common guidelines we should follow in order to prevent our systems to become compromised \cite{genernalmitigation}: 
\begin{itemize}
\item Keep the operating system and firewall up-to-date and patched.
\item Install robust anti-virus software and keep it updated, so it contains the latest malware definitions.
\item Create accounts for privilege user and normal user and limit permissions for unprivileged user when installing, updating, and uninstalling software.
\item Create regular backups of your operating system to revert back to in the case of infection.
\item Keep important files backed up in on an external storage device and offsite via an online storage service (e.g. Dropbox, OneDrive etc).
\end{itemize}


\section{Conclusions}
During this study we found Microsoft Windows is the operating systems that is targeted the most by ransomware. We believe that the reasons behind this correlate to the wide spread of Windows worldwide and to the wide variety of attack vectors and vulnerabilities found. While some ransomware attacks on Unix based systems exist~\cite{linuxRansomware}~\cite{osxRansomware} they aren't very well constructed as more often than not, the encryption is irreversible even after paying the ransom due to the lack of a remote channel of communication embedded in the binaries.

We also found that most ransomware make use of strong, proven cryptography making for a very effective attack which causes the user to be likely to pay the ransom if they value their data as there is no way of getting it back without the correct key.

When reading reports and articles detailing the misfortunes caused by recent attacks it becomes obvious that they usually exploit user naiveness and dated vulnerabilities which are often fixed before exploits are created but still work because users do not update their system as soon as it becomes available. This is an interesting psychological interaction between users and the Windows Update mechanism which is disliked by many because it starts at random times and hogs all of the machine's CPU. Microsoft has done some improvments that let users select at which time of day it is ok to update the system but nonetheless some users still defere critical updates because of it. In addition to the updating policies discussed in Section \ref{sec:mitigations} it is therefore crucial for companies and corporations to increase employee awareness with appropriate training that helps them recognize potential attack vectors.


% quote here: many of the ransomware attacks studied used trusted tools like Microsoft's PowerShell to slip past certain antivirus software. More prevalent malware even used PowerShell as a means to download and run the actual ransomware executable. --- Is this actually true? What i found online is that Ransom.PowerWare uses powershell but there is no mention of any of the ones we have studied using it :P https://www.symantec.com/content/dam/symantec/docs/security-center/white-papers/increased-use-of-powershell-in-attacks-16-en.pdf
%https://www.theguardian.com/technology/2017/may/15/windows-xp-patch-wannacry-ransomware-wecry-wanacrypt0r
% Use of proven cryptography
% out-of-date systems and software
% most users lack awareness of virus attack when browsing the Internet.

\newpage

\begin{thebibliography}{1}
\addcontentsline{toc}{section}{References}

\bibitem{trend}
Trend Micro
\textit{Ransomware}
\\\url{https://www.trendmicro.com/vinfo/us/security/definition/ransomware}

\bibitem{grown1}
Dunn, John E.
\textit{Ransom Trojans spreading beyond Russian heartland}, 09 March 2012
\\\url{https://www.techworld.com/news/security/ransom-trojans-spreading-beyond-russian-heartland-3343528/}

\bibitem{grown2}
FBI
\textit{New Internet scam: Ransomware}, 09 August 2012
\\\url{https://www.fbi.gov/news/stories/new-internet-scam/new-internet-scam}

\bibitem{mcafee}
Cyber criminals using Android malware and ransomware the most
\textit{McAfee}, 16 September 2013
\\\url{https://www.infoworld.com/article/2614854/security/update-mcafee--cyber-criminals-using-android-malware-and-ransomware-the-most.html}

\bibitem{bbc}
Cryptolocker victims to get files back for free
\textit{BBC News}, 06 August 2014
\\\url{http://www.bbc.co.uk/news/technology-28661463}

\bibitem{cyberwall}
FBI says crypto ransomware has raked in >\$18 million for cybercriminals
\textit{Ars Technica}, 25 June 2015
\\\url{https://arstechnica.com/information-technology/2015/06/fbi-says-crypto-ransomware-has-raked-in-18-million-for-cybercriminals}

\bibitem{symantec}
Dick O’Brien
\textit{Internet Security Threat Report: Ransomware}, July 2017
\\\url{https://www.symantec.com/content/dam/symantec/docs/security-center/white-papers/istr-ransomware-2017-en.pdf}

\bibitem{genernalmitigation}
Drive-By Downloads: How They Attack and How to Defend Yourself
\\\url{https://www.tomsguide.com/us/driveby-download,news-18329.html}

\bibitem{revetonremoval}
Trojan:W32/Reveton
\\\url{https://www.f-secure.com/v-descs/trojan_w32_reveton.shtml}

\bibitem{revetonanalysis}
The “Police Trojan”
AN IN-DEPTH ANALYSIS
\\\url{https://www.trendmicro.de/cloud-content/us/pdfs/security-intelligence/white-papers/wp_police_trojan.pdf}

\bibitem{torcircuits}
Tor: Hidden Service Protocol
\\\url{https://www.torproject.org/docs/hidden-services.html.en}

\bibitem{PEtoolkit}
pev, the PE file analysis toolkit
\\\url{http://pev.sourceforge.net}

\bibitem{pestrVideo}
WannaCry 2.0 - Three ways to find the Kill Switch
\\\url{https://www.youtube.com/watch?v=d56g3wahBck}

\bibitem{wanacryFindIpAddr}
WannaCry 2.0 Ransomware
\\\url{https://www.youtube.com/watch?v=d\_j8UUQbJsc}

\bibitem{wanacrySystems}
Almost all WannaCry victims were running Windows 7
\\\url{https://www.theverge.com/2017/5/19/15665488/wannacry-windows-7-version-xp-patched-victim-statistics}

\bibitem{linuxRansomware}
KillDisk now targeting Linux: Demands \$250K ransom, but can’t decrypt
\\\url{https://www.welivesecurity.com/2017/01/05/killdisk-now-targeting-linux-demands-250k-ransom-cant-decrypt/}

\bibitem{osxRansomware}
New OS X Ransomware KeRanger Infected Transmission BitTorrent Client Installer
\\\url{https://researchcenter.paloaltonetworks.com/2016/03/new-os-x-ransomware-keranger-infected-transmission-bittorrent-client-installer/}

\end{thebibliography}

\end{document}
